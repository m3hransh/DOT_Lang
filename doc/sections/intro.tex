\section{Introduction}
Domain Specific Languages are languages created to support
a particular set of tasks, as they are performed in a 
specific domain.Some DSLs are intended to be used by 
programmers, and therefore are more technical, while 
others are intended to be used by someone who is not 
a programmer and therefore they use less geeky concepts 
and syntax.

\par
Why using a specific, limited language instead of a 
generic, powerful one?
The short answer is that Domain Specific Languages 
are limited in the things they can do, but because 
of their specialization they can do much more in 
their own limited domain.
 
Domain Specific Languages are great because:
\begin{enumerate}
\item 
They let you communicate with domain experts. 
Do you write medical applications? Doctors do 
not understand when you talk about arrays or 
for-loops, but if you use a DSL that is about 
patients, temperature measures and blood pressure they could understand it better than you do
\item
They let you focus on the important concepts. 
They hide the implementation or the technical 
details and expose just the information that really matters.
\end{enumerate}

In this Report we are going to implement DOT language. 
DOT is a language that can describe graphs, either 
directed or non directed.

\subsection{What is ANTLR?}
ANTLR is a parser generator, a tool that helps you to 
create parsers
What you need to do to get an AST:
\begin{enumerate}
\item 
define a lexer and parser grammar
\item
invoke ANTLR: it will generate a lexer 
and a parser in your target language 
(e.g., Java, Python, C\#, Javascript)
\item
use the generated lexer and parser: you invoke 
them passing the code to recognize and they 
return to you an AST
\end{enumerate}
ANTLR is actually made up of two main parts: 
the tool, used to generate the lexer 
and parser, and the runtime, needed to run them.

\subsection{Lexers and Parsers}
A lexer takes the individual characters and 
transforms them in tokens, the atoms that the 
parser uses to create the logical structure.
These are notes that are important to define lexers
and parsers in ANTLR:

\begin{itemize}
    \item  lexer rules are all uppercase, while 
    parser rules are all lowercase
    \item  Rules are typically written in this 
    order: first the parser rules and then the lexer ones
    \item  lexer rules are analyzed in the order 
    that they appear(identifier define first)
    \item  The basic syntax of a rule is easy: 
    there is a name, a colon, the definition 
    of the rule and a terminating semicolon

\end{itemize}

In the Listing \ref{list:antlrex} an example of definition
 of lexer and parser in ANTLR syntax is shown.
This is not a complete grammar, but we can already 
see that lexer rules are all uppercase, while 
parser rules are all lowercase.
\lstinputlisting[style=ANTLR,caption={Example of defining parser and lexer in ANTLR},
    label={list:antlrex}]{codes/antlrex.txt}