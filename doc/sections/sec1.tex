\section{DOT Language}
The following (Listing \ref{list:dot-parser}) is the parser grammar 
defining the DOT language. First line indicates that these
are parser rules. Line 5 causes to use the DOTLexer file that is shown
in the Listing \ref{list:dot-lexer}. 
Following Rules are typical indications that are in regular expression
syntax. if you are not familiar with these rules you can click
\href{https://en.wikipedia.org/wiki/Regular_expression#Syntax}{\textbf{here.}}
Take note that, Terminals are shown in 
upper-case font and nonterminals in lower-case.
literal characters are given 
in single quotes. Parentheses \textbf{(} and \textbf{)} indicate 
grouping when needed. Question marks \textbf{?} 
represent optional items. Vertical bars \textbf{|} 
separate alternatives.

\lstinputlisting[style=ANTLR,
caption={Defining Parser for DOT Language in ANTLR},
    label={list:dot-parser}]{../src/antlr/DOTParser.g4}

Listing \ref{list:dot-lexer} defines the lexer rules. Fragmets 
are reusable building blocks for lexer rules that are support
case-insensatives keywords.
The keywords \textbf{node, edge, graph, digraph, 
subgraph,} and \textbf{strict} are case-independent. Note also that 
the allowed compass point values are not keywords, so these 
strings can be used elsewhere as ordinary identifiers and, 
conversely, the parser will actually accept any identifier.
a name is Any string of alphabetic 
([a-zA-Z\textbackslash 200-\textbackslash 377]) characters, 
underscores ('_') or digits ([0-9]), not beginning 
with a digit or any double-quoted string ("...") possibly 
containing escaped quotes (\textbackslash ") .An edgeop is -> in directed graphs 
and -- in undirected graphs.
The language supports C++-style 
comments: /* */ and //. In addition, a line beginning 
with a '\#' character is considered a line output from 
a C preprocessor (e.g., \# 34 to indicate line 34 ) and discarded.
Semicolons and commas aid readability but are not 
required. Also, any amount of whitespace may be inserted 
between terminals.
\lstinputlisting[style=ANTLR,
caption={Defining Lexer for DOT Language in ANTLR},
    label={list:dot-lexer}]{../src/antlr/DOTLexer.g4}

Figure \ref{fig:dot} shows an example of executing the 
language using 
\href{https://dot2tex.readthedocs.io/en/latest/}{\textbf{dot2tex}} 
console program.

\begin{figure}[H]
  \centering
\begin{tabular}{p{0.5\textwidth}p{0.3\textwidth}}
\subfloat[]{\lstinputlisting[style=mystyle]{codes/dot.dt}}&

\subfloat[]{\begin{tikzpicture}[anchor=mid,>=latex',scale=.6,line join=bevel,]
  \pgfsetlinewidth{1bp}
%%
\begin{scope}
  \pgfsetstrokecolor{black}
  \definecolor{strokecol}{rgb}{0.83,0.83,0.83};
  \pgfsetstrokecolor{strokecol}
  \definecolor{fillcol}{rgb}{0.83,0.83,0.83};
  \pgfsetfillcolor{fillcol}
  \filldraw (8.0bp,71.0bp) -- (8.0bp,362.0bp) -- (105.0bp,362.0bp) -- (105.0bp,71.0bp) -- cycle;
  \definecolor{strokecol}{rgb}{0.0,0.0,0.0};
  \pgfsetstrokecolor{strokecol}
  \draw (56.5bp,350.5bp) node {process \#1};
\end{scope}
\begin{scope}
  \pgfsetstrokecolor{black}
  \definecolor{strokecol}{rgb}{0.0,0.0,1.0};
  \pgfsetstrokecolor{strokecol}
  \draw (115.0bp,71.0bp) -- (115.0bp,362.0bp) -- (212.0bp,362.0bp) -- (212.0bp,71.0bp) -- cycle;
  \definecolor{strokecol}{rgb}{0.0,0.0,0.0};
  \pgfsetstrokecolor{strokecol}
  \draw (163.5bp,350.5bp) node {process \#2};
\end{scope}
  \pgfsetcolor{black}
  % Edge: a0 -> a1
  \draw [->] (68.0bp,294.7bp) .. controls (68.0bp,286.98bp) and (68.0bp,277.71bp)  .. (68.0bp,259.1bp);
  % Edge: a1 -> a2
  \draw [->] (67.753bp,222.7bp) .. controls (67.643bp,214.98bp) and (67.51bp,205.71bp)  .. (67.244bp,187.1bp);
  % Edge: a2 -> a3
  \draw [->] (67.247bp,150.7bp) .. controls (67.357bp,142.98bp) and (67.49bp,133.71bp)  .. (67.756bp,115.1bp);
  % Edge: a3 -> a0
  \draw [->] (53.874bp,112.86bp) .. controls (45.438bp,122.81bp) and (35.512bp,136.66bp)  .. (31.0bp,151.0bp) .. controls (16.598bp,196.79bp) and (17.573bp,213.22bp)  .. (32.0bp,259.0bp) .. controls (35.36bp,269.66bp) and (41.63bp,280.12bp)  .. (54.317bp,297.05bp);
  % Edge: b0 -> b1
  \draw [->] (153.15bp,295.05bp) .. controls (154.58bp,287.35bp) and (156.31bp,278.03bp)  .. (159.79bp,259.28bp);
  % Edge: b1 -> b2
  \draw [->] (166.39bp,223.05bp) .. controls (167.93bp,215.35bp) and (169.79bp,206.03bp)  .. (173.54bp,187.28bp);
  % Edge: b2 -> b3
  \draw [->] (170.6bp,151.41bp) .. controls (167.49bp,143.34bp) and (163.67bp,133.43bp)  .. (156.54bp,114.96bp);
  % Edge: a1 -> b3
  \draw [->] (78.824bp,224.49bp) .. controls (85.957bp,214.05bp) and (95.357bp,199.9bp)  .. (103.0bp,187.0bp) .. controls (115.4bp,166.07bp) and (128.21bp,141.58bp)  .. (141.98bp,114.34bp);
  % Edge: a3 -> end
  \draw [->] (77.099bp,79.688bp) .. controls (81.64bp,71.547bp) and (87.276bp,61.445bp)  .. (97.427bp,43.246bp);
  % Edge: b2 -> a3
  \draw [->] (158.01bp,155.81bp) .. controls (140.66bp,144.67bp) and (114.81bp,128.06bp)  .. (86.769bp,110.05bp);
  % Edge: b3 -> end
  \draw [->] (140.9bp,79.688bp) .. controls (136.36bp,71.547bp) and (130.72bp,61.445bp)  .. (120.57bp,43.246bp);
  % Edge: start -> a0
  \draw [->] (101.09bp,372.92bp) .. controls (95.714bp,363.34bp) and (88.442bp,350.4bp)  .. (77.023bp,330.07bp);
  % Edge: start -> b0
  \draw [->] (116.91bp,372.92bp) .. controls (122.29bp,363.34bp) and (129.56bp,350.4bp)  .. (140.98bp,330.07bp);
  % Node: a0
\begin{scope}
  \definecolor{strokecol}{rgb}{1.0,1.0,1.0};
  \pgfsetstrokecolor{strokecol}
  \definecolor{fillcol}{rgb}{1.0,1.0,1.0};
  \pgfsetfillcolor{fillcol}
  \filldraw [opacity=1] (68.0bp,313.0bp) ellipse (27.0bp and 18.0bp);
  \definecolor{strokecol}{rgb}{0.0,0.0,0.0};
  \pgfsetstrokecolor{strokecol}
  \draw (68.0bp,313.0bp) node {a0};
\end{scope}
  % Node: a1
\begin{scope}
  \definecolor{strokecol}{rgb}{1.0,1.0,1.0};
  \pgfsetstrokecolor{strokecol}
  \definecolor{fillcol}{rgb}{1.0,1.0,1.0};
  \pgfsetfillcolor{fillcol}
  \filldraw [opacity=1] (68.0bp,241.0bp) ellipse (27.0bp and 18.0bp);
  \definecolor{strokecol}{rgb}{0.0,0.0,0.0};
  \pgfsetstrokecolor{strokecol}
  \draw (68.0bp,241.0bp) node {a1};
\end{scope}
  % Node: a2
\begin{scope}
  \definecolor{strokecol}{rgb}{1.0,1.0,1.0};
  \pgfsetstrokecolor{strokecol}
  \definecolor{fillcol}{rgb}{1.0,1.0,1.0};
  \pgfsetfillcolor{fillcol}
  \filldraw [opacity=1] (67.0bp,169.0bp) ellipse (27.0bp and 18.0bp);
  \definecolor{strokecol}{rgb}{0.0,0.0,0.0};
  \pgfsetstrokecolor{strokecol}
  \draw (67.0bp,169.0bp) node {a2};
\end{scope}
  % Node: a3
\begin{scope}
  \definecolor{strokecol}{rgb}{1.0,1.0,1.0};
  \pgfsetstrokecolor{strokecol}
  \definecolor{fillcol}{rgb}{1.0,1.0,1.0};
  \pgfsetfillcolor{fillcol}
  \filldraw [opacity=1] (68.0bp,97.0bp) ellipse (27.0bp and 18.0bp);
  \definecolor{strokecol}{rgb}{0.0,0.0,0.0};
  \pgfsetstrokecolor{strokecol}
  \draw (68.0bp,97.0bp) node {a3};
\end{scope}
  % Node: b0
\begin{scope}
  \definecolor{strokecol}{rgb}{0.0,0.0,0.0};
  \pgfsetstrokecolor{strokecol}
  \definecolor{fillcol}{rgb}{0.83,0.83,0.83};
  \pgfsetfillcolor{fillcol}
  \filldraw [opacity=1] (150.0bp,313.0bp) ellipse (27.0bp and 18.0bp);
  \draw (150.0bp,313.0bp) node {b0};
\end{scope}
  % Node: b1
\begin{scope}
  \definecolor{strokecol}{rgb}{0.0,0.0,0.0};
  \pgfsetstrokecolor{strokecol}
  \definecolor{fillcol}{rgb}{0.83,0.83,0.83};
  \pgfsetfillcolor{fillcol}
  \filldraw [opacity=1] (163.0bp,241.0bp) ellipse (27.0bp and 18.0bp);
  \draw (163.0bp,241.0bp) node {b1};
\end{scope}
  % Node: b2
\begin{scope}
  \definecolor{strokecol}{rgb}{0.0,0.0,0.0};
  \pgfsetstrokecolor{strokecol}
  \definecolor{fillcol}{rgb}{0.83,0.83,0.83};
  \pgfsetfillcolor{fillcol}
  \filldraw [opacity=1] (177.0bp,169.0bp) ellipse (27.0bp and 18.0bp);
  \draw (177.0bp,169.0bp) node {b2};
\end{scope}
  % Node: b3
\begin{scope}
  \definecolor{strokecol}{rgb}{0.0,0.0,0.0};
  \pgfsetstrokecolor{strokecol}
  \definecolor{fillcol}{rgb}{0.83,0.83,0.83};
  \pgfsetfillcolor{fillcol}
  \filldraw [opacity=1] (150.0bp,97.0bp) ellipse (27.0bp and 18.0bp);
  \draw (150.0bp,97.0bp) node {b3};
\end{scope}
  % Node: end
\begin{scope}
  \definecolor{strokecol}{rgb}{0.0,0.0,0.0};
  \pgfsetstrokecolor{strokecol}
  \draw (130.5bp,43.0bp) -- (87.5bp,43.0bp) -- (87.5bp,0.0bp) -- (130.5bp,0.0bp) -- cycle;
  \draw (99.5bp,43.0bp) -- (87.5bp,31.0bp);
  \draw (87.5bp,12.0bp) -- (99.5bp,0.0bp);
  \draw (118.5bp,0.0bp) -- (130.5bp,12.0bp);
  \draw (130.5bp,31.0bp) -- (118.5bp,43.0bp);
  \draw (109.0bp,21.5bp) node {end};
\end{scope}
  % Node: start
\begin{scope}
  \definecolor{strokecol}{rgb}{0.0,0.0,0.0};
  \pgfsetstrokecolor{strokecol}
  \draw (109.0bp,406.0bp) -- (62.25bp,388.0bp) -- (109.0bp,370.0bp) -- (155.75bp,388.0bp) -- cycle;
  \draw (73.44bp,392.31bp) -- (73.44bp,383.69bp);
  \draw (97.8bp,374.31bp) -- (120.2bp,374.31bp);
  \draw (144.56bp,383.69bp) -- (144.56bp,392.31bp);
  \draw (120.2bp,401.69bp) -- (97.8bp,401.69bp);
  \draw (109.0bp,388.0bp) node {start};
\end{scope}
%
\end{tikzpicture}
% End of code
}
\end{tabular}
  
  \caption{Example of executing DOT Language uisng dot2tex console program}
  \label{fig:dot}
\end{figure}

\subsection{Subgraphs and Clusters}
Subgraphs play two roles. First, a subgraph 
can be used to represent graph structure, 
indicating that certain nodes and edges should 
be grouped together. This is the usual role for 
subgraphs and typically specifies semantic 
information about the graph components. It 
can also provide a convenient shorthand for 
edges. An edge statement allows a subgraph on 
both the left and right sides of the edge operator. 
When this occurs, an edge is created from every 
node on the left to every node on the right. 
For example, the specification
\begin{lstlisting}[language=C, style=mystyle]
  A -> {B C}
\end{lstlisting}
is equivalent to:
\begin{lstlisting}[language=C, style=mystyle]
  A -> B 
  A -> C
\end{lstlisting}
In the second role, a subgraph can provide a 
context for setting attributes. For example, a 
subgraph could specify that blue is the default 
color for all nodes defined in it. In the context 
of graph drawing, a more interesting example is:
\begin{lstlisting}[language=Java, style=mystyle]

subgraph { 
rank = same; A; B; C; 
} 
\end{lstlisting}
This (anonymous) subgraph specifies that the nodes 
A, B and C should all be placed on the same rank.

\subsection{Lexical and Semantic Notes}
A graph must be specified as either a digraph or a 
graph. Semantically, this indicates whether or not 
there is a natural direction from one of the edge's 
nodes to the other. Lexically, a digraph must specify 
an edge using the edge operator -> while a undirected 
graph must use --. Operationally, the distinction 
is used to define different default rendering 
attributes. For example, edges in a digraph will 
be drawn, by default, with an arrowhead pointing 
to the head node. For ordinary graphs, edges are 
drawn without any arrowheads by default.

A graph may also be described as strict. This 
forbids the creation of multi-edges, i.e., there 
can be at most one edge with a given tail node and 
head node in the directed case. For undirected 
graphs, there can be at most one edge connected to 
the same two nodes. Subsequent edge statements 
using the same two nodes will identify the edge 
with the previously defined one and apply any 
attributes given in the edge statement. For 
example, the graph
\begin{lstlisting}[language=C, style=mystyle]
strict graph { 
  a -- b
  a -- b
  b -- a [color=blue]
} 
\end{lstlisting}
will have a single edge connecting nodes a and b, 
whose color is blue.

If a default attribute is defined using a node, 
edge, or graph statement, or by an attribute 
assignment not attached to a node or edge, any 
object of the appropriate type defined afterwards 
will inherit this attribute value. This holds until 
the default attribute is set to a new value, from 
which point the new value is used. Objects defined 
before a default attribute is set will have an 
empty string value attached to the attribute once 
the default attribute definition is made.

Note, in particular, that a subgraph receives the 
attribute settings of its parent graph at the time 
of its definition. This can be useful; for example, 
one can assign a font to the root graph and all 
subgraphs will also use the font. For some 
attributes, however, this property is undesirable. 
If one attaches a label to the root graph, it is 
probably not the desired effect to have the label 
used by all subgraphs. Rather than listing the 
graph attribute at the top of the graph, and the 
resetting the attribute as needed in the subgraphs, 
one can simply defer the attribute definition in 
the graph until the appropriate subgraphs have been 
defined.

If an edge belongs to a cluster, its endpoints 
belong to that cluster. Thus, where you put an 
edge can effect a layout, as clusters are sometimes 
laid out recursively.

There are certain restrictions on subgraphs 
and clusters. First, at present, the names of a 
graph and it subgraphs share the same namespace. 
Thus, each subgraph must have a unique name. 
Second, although nodes can belong to any number 
of subgraphs, it is assumed clusters form a strict 
hierarchy when viewed as subsets of nodes and edges.

